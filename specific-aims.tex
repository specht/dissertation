% ==============================================================
\chapter{Specific aims}
% ==============================================================

The vast amounts of data acquired in large-scale mass spectrometric experiments
require software systems which are powerful enough to perform data evaluation
in a way which is fast enough and does not represent a potential bottleneck in 
the experimental data evaluation workflow.
Therefore, the available systems which provide mass-spectrometric data 
evaluation platforms to researches require further improvement.
Most importantly, a decentralized setup which supports all common types of 
operating systems allows for higher throughput at minimal cost because commodity
hardware can be employed.
Furthermore, the utilization of the various software tools needs to be further 
facilitated.

With such a decentralized MS/MS data evaluation system available, the
characterization of the chloroplast proteome in \cre~will be performed using
semi-quantitative analysis.
Furthermore, an analysis of the anaerobic response of the chloroplast proteome
will be carried out, employing SILAC as a labeling strategy.
Specialized software for the high-throughput analysis of SILAC samples will
be presented in this thesis.

The existing implementation of the Genomic Peptide Finder needs to
be improved in terms of search speed and sensitivity.
Another issue to be resolved is the statistical validation of GPF peptide 
assignments to MS/MS spectra.
Finally, the question arises whether peptides identified via {\em de novo} prediction
and subsequent matching via the Genomic Peptide Finder can be used to support 
the genome annotation of \cre~in the spirit of proteogenomics.
% It will be shown in this thesis that GPF-supported proteogenomic annotation 
% using high-throughput mass spectrometric data is feasible.
