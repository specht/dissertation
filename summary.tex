\cleardoublepage
% ==============================================================
\chapter{Summary}
% ==============================================================

A novel platform for the evaluation of mass spectrometric data has been
established and was subsequently employed for the evaluation of measurements
performed to characterize the chloroplast proteome of \cre~and elucidate
the anaerobic response of the chloroplast proteome.
Proteomatic provides user-friendly access to mass spectrometric data 
evaluation programs and allows for the construction of complex workflows.

In order to facilitate the large-scale characterization of the chloroplast
proteome, qTrace has been designed to automate the peptide and protein 
quantitation process in full scans.
qTrace supports a variety of metabolic labeling strategies and has 
confirmed the anaerobic induction of proteins previously shown to be induced on
the transcript level.
Furthermore, several proteins of unknown function have been found induced
under anaerobic conditions which represent interesting targets for further 
research.

{\em De novo}-based peptide sequencing is a powerful approach for the annotation
of MS/MS spectra, complementing gene model-supported database search algorithms.
The inherent ambiguity of {\em de novo} sequenced peptides is greatly diminished
by the Genomic Peptide Finder which matches the potentially erroneous
peptide sequences to the genomic DNA sequence of an organism, thereby discarding
spurious peptides and correcting erroneous {\em de novo} sequences.
A new version of the Genomic Peptide Finder, improved in terms of sensitivity, 
specificity, and search speed has been presented in this thesis.

Furhermore, a method for the automated statistical validation of GPF peptides
has been developed and has led to the possibility of using GPF peptides as
an extrinsic hint source for AUGUSTUS in a high-throughput proteogenomic genome 
annotation pipeline.

All software developed in the scope of this thesis is publicly available
at \href{http://github.com/specht}{http://github.com/specht}.
