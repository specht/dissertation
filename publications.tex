% ==============================================================
\chapter{Publications}
% ==============================================================

The following publications are presented in this thesis:

\begin{enumerate}
\item 
{\bf Proteomics to go: Proteomatic enables the user-friendly creation of versatile MS/MS data evaluation workflows.}

Michael Specht, Sebastian Kuhlgert, Christian Fufezan and Michael Hippler

Bioinformatics 2011; doi: 10.1093/bioinformatics/btr081 (in press).

\item
{\bf Characterizing the anaerobic response of Chlamydomonas reinhardtii by quantitative proteomics.}

Mia Terashima, Michael Specht, Bianca Naumann-Busch and Michael Hippler

Mol Cell Proteomics, 9(7), 2010: 1514-32.

\item
{\bf The chloroplast proteome: A survey form the {\em Chlamydomonas reinhardtii} perspective with a focus on distinctive features.}

Mia Terashima, Michael Specht and Michael Hippler

2011, in minor revision.

\item
{\bf Concerted action of the new Genomic Peptide Finder and AUGUSTUS allows for automated proteogenomic annotation of the {\em Chlamydomonas reinhardtii} genome.}

Michael Specht, Mario Stanke, Mia Terashima, Bianca Naumann-Busch, Ingrid Jan\"sen, Ricarda H\"ohner, Erik F.~Y.~Hom, Chun Liang and Michael Hippler

Proteomics (2011), in press.

% \item
% {\bf T. oceanica paper with Markus Lommer}
% 
\item
{\bf p3d -- Python module for structural bioinformatics}

Christian Fufezan and Michael Specht

BMC Bioinformatics (2009) 10:258.

\end{enumerate}

% --------------------------------------------------------------
\cleardoublepage
\section{Proteomics to go: Proteomatic enables the user-friendly creation of versatile MS/MS data evaluation workflows}
% \markboth{Proteomics to go: Proteomatic enables the user-friendly creation of versatile MS/MS data evaluation workflows}{Proteomics to go: Proteomatic enables the user-friendly creation of versatile MS/MS data evaluation workflows}
% \addcontentsline{toc}{section}{Proteomics to go: Proteomatic enables the user-friendly creation of versatile MS/MS data evaluation workflows}

Michael Specht, Sebastian Kuhlgert, Christian Fufezan and Michael Hippler

Bioinformatics 2011; doi: 10.1093/bioinformatics/btr081 (in press).

\subsection*{Contributions}

\begin{itemize}
\item design and implementation of Proteomatic
\item main contribution to the text
\item creation of figures
\item corresponding authorship
\end{itemize}

\includepublication{publications/proteomatic-2011.pdf}

% --------------------------------------------------------------
\cleardoublepage
\section{Characterizing the anaerobic response of {\em Chlamydomonas reinhardtii} by quantitative proteomics}
% \markboth{Characterizing the anaerobic response of {\em Chlamydomonas reinhardtii} by quantitative proteomics}{Characterizing the anaerobic response of {\em Chlamydomonas reinhardtii} by quantitative proteomics}
% \addcontentsline{toc}{section}{Characterizing the anaerobic response of {\em Chlamydomonas reinhardtii} by quantitative proteomics}
% --------------------------------------------------------------

\subsection*{Contributions}

\begin{itemize}
\item what was it, then?
\end{itemize}

\includepublication{publications/terashima-2010.pdf}

% --------------------------------------------------------------
\cleardoublepage
\section{The chloroplast proteome: A survey form the {\em Chlamydomonas reinhardtii} perspective with a focus on distinctive features}
% \markboth{The chloroplast proteome: A concise survey form the {\em Chlamydomonas reinhardtii} perspective}{The chloroplast proteome: A concise survey form the {\em Chlamydomonas reinhardtii} perspective}
% \addcontentsline{toc}{section}{The chloroplast proteome: A concise survey form the {\em Chlamydomonas reinhardtii} perspective}
% --------------------------------------------------------------

\subsection*{Contributions}

\begin{itemize}
\item what was it, then?
\end{itemize}

\includepublication{publications/terashima-2011-ms.pdf}

% --------------------------------------------------------------
\cleardoublepage
\section{Concerted action of the new Genomic Peptide Finder and AUGUSTUS allows for automated proteogenomic annotation of the {\em Chlamydomonas reinhardtii} genome}
% \markboth{Concerted action of the new Genomic Peptide Finder and AUGUSTUS allows for automated proteogenomic annotation of the {\em Chlamydomonas reinhardtii} genome}{Concerted action of the new Genomic Peptide Finder and AUGUSTUS allows for automated proteogenomic annotation of the {\em Chlamydomonas reinhardtii} genome}
% \addcontentsline{toc}{section}{Concerted action of the new Genomic Peptide Finder and AUGUSTUS allows for automated proteogenomic annotation of the {\em Chlamydomonas reinhardtii} genome}
% --------------------------------------------------------------

\label{paper:gpf}

\subsection*{Contributions}

\begin{itemize}
\item what was it, then?
\end{itemize}

\includepublication{publications/gpf-2011.pdf}

% % --------------------------------------------------------------
% \cleardoublepage
% \section{T.~oceanica paper with Markus Lommer}
% % \markboth{T.~oceanica paper with Markus Lommer}{T.~oceanica paper with Markus Lommer}
% % \addcontentsline{toc}{section}{T.~oceanica paper with Markus Lommer}
% % --------------------------------------------------------------
% 
% \subsection*{Contributions}
% 
% \begin{itemize}
% \item what was it, then?
% \end{itemize}
% 
% % \includepublication{publications/lommer-2011.pdf}

% --------------------------------------------------------------
\cleardoublepage
\section{p3d -- Python module for structural bioinformatics}
% \markboth{p3d -- Python module for structural bioinformatics}{p3d -- Python module for structural bioinformatics}
% \addcontentsline{toc}{section}{p3d -- Python module for structural bioinformatics}
% --------------------------------------------------------------

\subsection*{Contributions}

\begin{itemize}
\item what was it, then?
\end{itemize}

\includepublication{publications/fufezan-2009.pdf}
