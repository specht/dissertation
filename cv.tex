% ==============================================================
\chapter*{Curriculum vitae}
\markboth{Curriculum vitae}{Curriculum vitae}
\addcontentsline{toc}{chapter}{Curriculum vitae}
% ==============================================================

% Michael Specht \\
% Toppheideweg 34 \\
% 48161 Münster
% 
% Phone: +49 251 4808158 \\
% E-mail: \href{mailto:michael.specht@uni-muenster.de}{michael.specht@uni-muenster.de}
% 

\begin{longtable}{@{}lp{12.5cm}}

\cvsubheader{Personal details}

Date of birth: & December 23, 1981 \\
Place of birth: & Magdeburg, Germany \\
Marital status: & married, two children\\
\\

\cvsubheader{Publications}

04/2011 & Terashima M., {\bf Specht M.}, Hippler M. (2011). The chloroplast proteome: A survey from the {\em Chlamydomonas reinhardtii} perspective with a focus on distinctive features. Current Genetics 2011 (in press). \\%; DOI: \href{http://dx.doi.org/}{} (in press). \\

03/2011 & {\bf Specht M.}, Stanke M., Terashima M., Naumann-Busch B., Janßen I., Höhner R., Hom E. F. Y., Liang C., Hippler M. (2011). Concerted action of the new Genomic Peptide Finder and AUGUSTUS allows for automated proteogenomic annotation of the Chlamydomonas reinhardtii genome. Proteomics 2011; DOI: \href{http://dx.doi.org/10.1002/pmic.201000621}{10.1002/pmic.201000621} (in press). \\

02/2011 & {\bf Specht M.}, Kuhlgert S., Fufezan C., Hippler M. (2011). Proteomics to go: Proteomatic enables the user-friendly creation of versatile MS/MS data evaluation workflows. Bioinformatics 2011; DOI: \href{http://dx.doi.org/10.1093/bioinformatics/btr081}{10.1093/bioinformatics/btr081} (in press). \\

07/2010 & Terashima M., {\bf Specht M.}, Naumann B., Hippler M. (2010). Characterizing the anaerobic response of Chlamydomonas reinhardtii by quantitative proteomics. Mol Cell Proteomics, 9(7), 1514-32. \\

08/2009 & Fufezan C., {\bf Specht M.} (2009). p3d – Python module for structural bioinformatics. BMC Bioinformatics, 10:258. \\

11/2007 & Ropinski T., {\bf Specht M.}, Meyer-Spradow J., Hinrichs K., Preim B. (2007). Surface Glyphs for Visualizing Multimodal Volume Data. Vision, Modelling and Visualization (VMV) (3-13), Saarbrücken, 2007. \\

\newpage 

\cvsubheader{Talks}

\cvtitle{03/2011}{DGMS 2011, Dortmund}
& Proteomics to go: Proteomatic enables the user-friendly creation of versatile MS/MS data evaluation workflows. \\
\tabspace\\


\cvsubheader{Professional experience}

\cvtitle{since 04/2007}{Institute of Plant Biology and Biotechnology\newline Westfälische Wilhelms-Universität Münster}
& Doctoral student in the lab of Prof. Dr. Michael Hippler \newline
% \vspace{6pt}
\vspace{-9pt}
\begin{compactitem}
\item Proteomatic: design and implementation of a user-friendly, decentralized MS/MS data 
evaluation platform \vspace{4pt}\newline
Link: \href{http://www.proteomatic.org}{http://www.proteomatic.org}
% \vspace{-12pt}
% \begin{compactitem}
\item Genomic Peptide Finder: Software for the alignment of MS/MS {\em de novo}
predicted amino acid sequences to the genomic DNA sequence of an organism.
Resulting peptides may be used for evidence-based proteogenomic genome annotation. \vspace{4pt}\newline
Link: \href{http://github.com/specht/gpf}{http://github.com/specht/gpf}.
\item qTrace: Software for the high-throughput quantitation of metabolically
labeled samples (e.g. $^{\textrm{13}}$C Arg SILAC or $^{\textrm{15}}$N) in survey scans.\vspace{4pt}\newline
Link: \href{http://github.com/specht/qtrace}{http://github.com/specht/qtrace}
% \end{compactitem}
\item establishment of free MS/MS data evaluation software in the lab, resulting 
in increased throughput, lower costs and proliferation of MS/MS data 
evaluation-related expertise
\end{compactitem}
% \vspace{-6pt}
\tabspace\\

\cvtitle{10/2006 -- 03/2007}{PROVISIO Software GmbH Münster}
& Software developer -- Realtime Rendering Group \newline
\vspace{-9pt}
\begin{compactitem}
\item design and implementation of a fast JPEG decoder
\item implementation of various user interface widgets for Pictomio
(photo viewer software)
\end{compactitem}
\vspace{-6pt}
\tabspace\\

\cvtitle{07/2005 -- 02/2006}{1komma6 Multimediale Dienstleistungen GmbH Münster}
& Internship -- Web development, focus on accessibility \newline
\tabspace\\

\newpage

\cvsubheader{Education}
\cvtitle{03/2006 -- 11/2006}{Diploma thesis}
& \emph{Glyph-enhanced Volume Visualization}. \newline
Development of a visualization method for the interactive, simultaneous display 
of multiple, related medical data sets (CT and PET).\tabspace\\
% % \input{../../common-en/chromacoding}
% % \input{../../common-en/studienarbeit}
% 

\cvtitle{10/2001 -- 11/2006}{Study of Computational Visualistics}
& Study of Computational Visualistics (computer science with a focus on computer 
graphics) at the Otto-von-Guericke-Universität Magdeburg.\tabspace\\

\cvtitle{09/2000 -- 07/2001}{Alternative civilian service}
& Civilian service at the retirement home "`Luisenhaus"', Potsdam. \tabspace\\

\cvtitle{09/1992 -- 05/2000}{Secondary school}
& Werner-von-Siemens-Gymnasium, Magdeburg. \tabspace\\

% 
% \newpage
% 
\cvsubheader{Further education}
% 
\cvtitle{10/2010}{Advanced Scientific Programming in Python}
& Participated in an autumn school, held in Trento, Italy,
organized by G-Node, the center for Mind/Brian Sciences, and Fondazione Bruno Kessler.\tabspace\\

\cvsubheader{Miscellaneous projects}
% 
% % \input{../../common-en/nprflow}
% % \input{../../common-en/hamster}
% % \input{../../common-en/tutorium}
% % \input{../../common-en/vib}

\cvtitle{02/2003}{Student anti war group}
& Foundation of the student anti war group at the University of Magdeburg,
shortly before the beginning of the Iraq war in March 2003.
Planning and organization of discussion rounds and movie nights. \tabspace\\

\cvtitle{11/2002 -- 11/2003}{Short film "`pixelle ma belle \#01"'}
& \begin{compactitem}
\vspace{-9pt} 
\item design and implementation of a distributed video rendering software
\item development of an alternative, low-cost blue screen method
\end{compactitem}
\vspace{6pt} 
Link: \href{http://www.pixellemabelle.de}{www.pixellemabelle.de}
\tabspace\\

% % \input{../../common-en/fx}
% % \input{../../common-en/ferienlager}
% % \input{../../common-en/zeitnahme}
% 
\cvsubheader{Skills}
% 

foreign languages
& English (fluent)\newline
French (basic knowlegde) \tabspace\\

programming
& C, C+\kern-0.2em +, Qt, Ruby, Python, PHP, Java, Assembler, XHTML, CSS, Ajax, MySQL\tabspace\\

software
& Linux, Windows, Mac OS X \newline
Visual Studio, Microsoft Office, OpenOffice, \LaTeX, Photoshop, GIMP, Inkscape \newline
CVS, Subversion, Git\tabspace\\

% 
\end{longtable}
% 
